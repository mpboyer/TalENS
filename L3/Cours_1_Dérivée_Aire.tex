\documentclass{beamercours}

\title{Cours TalENS 2023-2024}
\subtitle{Dérivée, Volume, Aire, Périmètre}
\date{chaipakan}

\newtheorem{definitionfr}{Définition}


\begin{document}

\maketitle

    \section{Rappels Mathématiques}
        \subsection{Dérivation}
            \begin{frame}{Dérivée par rapport à une variable}
                
            \end{frame}

        \subsection{Polygones Réguliers et Solides Euclidiens}

    \section{Constatations}
        \subsection{En Dimension 2 : Le Cercle}
            \begin{frame}{Rayon, Périmètre, Aire}

            \end{frame}

        \subsection{En Dimension 3 : La Sphère}

        \subsection{Presque Contre-Exemples}
            \begin{frame}{Le Carré}
                
            \end{frame}

            \begin{frame}{Le Triangle Equilatéral}
                
            \end{frame}

            \begin{frame}{Les Polygones Réguliers}
                
            \end{frame}

            \begin{frame}{Le Cube}
                
            \end{frame}

    \section{Généralisation}
        \subsection{L'Aire et le Volume en $n$ Dimensions}
            \begin{frame}{Un Espace en $n$ Dimensions ?}
                
            \end{frame}

            \begin{frame}{Un Solide en $n$ Dimensions}
                
            \end{frame}

            \begin{frame}{Aire et Volume d'un Solide en $n$ Dimensions}
                
            \end{frame}
        \subsection{Relation entre Volume et Aire en $n$ Dimensions pour un Solide}

        \subsection{Et pour une forme quelconque ?}




\end{document}
