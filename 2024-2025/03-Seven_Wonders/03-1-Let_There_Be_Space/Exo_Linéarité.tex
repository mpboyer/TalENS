\documentclass{classe}
\title{Exercice Linéarité}
\author{Clément Allard --- Matthieu Boyer}
\date{18 janvier 2025}

\usepackage[pdftex,outline]{contour}
\renewcommand{\question}[1]{\underline{Question #1 :}}

\begin{document}

\section{Un sous espace vectoriel...}

On note $E = \mathbb{R}^3$ et $+$, $\cdot$ les opérations usuelles qui font de $E$ un $\mathbb{R}$ espace vectoriel. On note $F$ l'espace vectoriel engendré par les vecteurs $(1, 1, 0)$, $(0, 0, 1)$ et $(1, 1, 1)$ de $\mathbb{R}^3$.
\newline
\newline
\question{1} Est ce que la famille formée par ces trois vecteurs est libre ?
\newline
\newline
\question{2} Trouver une condition pour qu'un vecteur $(x, y, z)$ soit dans $F$.
\newline
\newline
\question{3} Quelle est la dimension de $F$ ?

\section{...et une application qui agit dessus}

On note $f$ l'application qui agit sur un vecteur de $F$ en effectuant la rotation d'angle $\pi$ d'axe $(Oz)$
\newline
\newline
\question{4} Déterminer l'image d'un vecteur de $F$ par $f$.
\newline
\newline
\question{5} Déterminer l'image de $f$, notée $\mathrm{Im}(f)$ définie de la manière suivante :
\begin{equation*}
y\in \mathrm{Im}(f) \iff \exists x \in F, y=f(x)
\end{equation*}
\question{6} Montrer que l'image de $f$ est un espace vectoriel. Quel est cet espace vectoriel ?

\end{document}