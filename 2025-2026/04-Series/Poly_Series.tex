\documentclass[info, math]{mpb-cours}

\title{Séries et Géométrie}
\author{Matthieu Boyer}
\date{Cours TalENS 4 2025-2026}

\begin{document}
\bettertitle
L'objectif de ce cours va être d'étudier une classe d'objets: les séries numériques réelles.

\section{Notion de Série}
\subsection{Des suites aux séries}
\begin{definition}
	Une \define{série} (réelle) est une suite réelle définie à partir d'une autre suite: la série de terme
	général $u_{n}$ est la suite de terme général $\sum_{k = 0}^{n} u_{k}$, somme des $n$ premiers termes de
	la suite $u_{n}$.
\end{definition}

\begin{proposition}
	L'ensemble des séries et l'ensemble des suites sont égaux.
\end{proposition}
\begin{proof}
	En effet, on peut définir toute suite de terme général $u_{n}$ par la série de terme général $u_{k} - u_{k - 1}$ pour $k \geq 1$ et de premier terme $u_{0}$.
\end{proof}

\begin{definition}
	Pour $S, S'$ deux séries de deux séries de terme généraux $a_{n}$ et $b_{n}$, on définit
	\begin{itemize}
		\item La somme $S + S'$ comme la somme des suites sous-jacentes (de manière équivalente, la série dont
		      le terme général est la somme des termes généraux de $S$ et $S'$);
		\item Le produit $\lambda S$ de $S$ par $\lambda$ comme le produit par $\lambda$ de la suite
		      sous-jacente (de manière équivalente, la série dont le terme général est le produit par $\lambda$
		      du terme général de $S$);
		\item Le produit $S\times S'$ comme la série de terme général
		      \begin{equation*}
			      c_{n} = \sum_{k = 0}^{n}a_{k}b_{n - k}.
		      \end{equation*}
	\end{itemize}
\end{definition}

\subsection{Convergence absolue ou non}
\begin{definition}
	Une série \define{converge} si et seulement si la suite sous-jacente converge.

	Une série de terme général $u_{n}$ \define{converge absolument} si et seulement si la série de terme
	général $\abs{u_{n}}$ converge.

	Une série est dite \define{semi-convergente} si elle est convergente mais pas absolument convergente.
\end{definition}

\begin{proposition}
	La série de terme général $\frac{1}{n}$ diverge.
\end{proposition}

\begin{definition}
	Si une série $S$ de terme général $u_{n}$ converge, on note $S = \sum_{n = 0}^{\infty} u_{n}$.
\end{definition}

\begin{proposition}
	Soient $S$ et $S'$ deux séries convergentes. Alors
	\begin{itemize}
		\item $S + S'$ converge vers $S + S'$;
		\item $\lambda S$ converge vers $\lambda S$ pour tout $\lambda \in \R$;
		\item $SS'$ converge vers $SS'$.
	\end{itemize}
\end{proposition}

Si $u_{n}$ et $v_{n}$ définissent deux suites, alors on note:
\begin{itemize}
	\item $u_{n} = \O(v_{n})$ quand $u_{n} \leq Cv_{n}$ pour $C \geq 0$ et $n$ suffisamment grand;
	\item $u_{n} \sim v_{n}$ quand $\frac{u_{n}}{v_{n}} \xrightarrow[n \to \infty]{} 1$.
\end{itemize}

\begin{proposition}
	Soient $S$ et $S'$ deux séries de termes généraux $u_{n}$ et $v_{n}$ respectivement:
	\begin{itemize}
		\item Si $u_{n} = \O(v_{n})$, alors la convergence de $S'$ implique celle de $S$;
		\item Si $u_{n} \sim v_{n}$ alors $S$ et $S'$ ont même nature.
	\end{itemize}
\end{proposition}

\begin{proposition}[Comparaison Série-Intégrale]
	La série de terme général $f(n)$ converge si et seulement si l'intégrale $\int_{0}^{+\infty} f$ converge.
\end{proposition}

\section{Exemples et Calculs}
\subsection{Séries exprimables}
\begin{definition}
	La \define{série géométrique} de paramètre $q$ est la série de terme général $q^{n}$.
\end{definition}

\begin{proposition}
	La série géométrique de paramètre $q$ converge vers $\frac{1}{1 - q}$ si et seulement si $\abs{q} < 1$.
\end{proposition}
\begin{proof}
	On a un télescopage
	\begin{equation*}
		\sum_{n = 0}^{N} q^{n} = \frac{1 - q^{N + 1}}{1 - q},
	\end{equation*}
	d'où le résultat.
\end{proof}

\begin{definition}
	La série de terme général $\frac{x^{n}}{n!}$ est appelée \define{série exponentielle}, est absolument
	convergente et on note $e^{x}$ sa limite.
\end{definition}

\subsection{Interlude sur Zeta}
\begin{definition}
	On définit $\zeta(s) = \sum_{n = 0}^{\infty} \frac{1}{n^{s}}$. C'est une fonction bien définie et
	dérivable à l'infini pour $s > 1$.
\end{definition}
\begin{proof}
	On utilise le théorème de comparaison série-intégrale pour obtenir le critère de Riemann.
\end{proof}

\begin{proposition}
	On a $\zeta(2) = \frac{\pi^{2}}{6}$.
\end{proposition}

\subsection{Séries de Taylor}
\begin{definition}
	Soit $f$ une fonction dérivable à l'infini en $0$. Sa \define{$n$-ème approximation de Taylor} en $0$ est
	donnée par
	\begin{equation*}
		f_{n}(x) = \sum_{k = 0}^{n} \frac{f^{(k)}(0)}{k!}x^{k}.
	\end{equation*}
	On dit que $f$ est \define{définissable en série de Taylor} si
	\begin{equation*}
		f(x) = \sum_{n = 0}^{\infty} \frac{f^{(n)}(0)}{n!}x^{n}.
	\end{equation*}
\end{definition}

\begin{proposition}
	Les fonctions $\exp, \ln(1 + x), \cos, \sin, \arccos, \arcsin$ et $\arctan$
	sont définissables en série de Taylor en $0$, avec
	\begin{itemize}
		\item $\exp x = \sum \frac{x^{n}}{n!}$;
		\item $\ln (1 + x) = \sum \frac{(-1)^{n}}{n}x^{n}$;
		\item $\cos x = \sum \frac{(-1)^{n}x^{2n}}{(2n)!}$;
		\item $\sin x = \sum \frac{(-1)^{n}x^{2n + 1}}{(2n + 1)!}$;
		\item $\arctan x = \sum (-1)^{n}\frac{x^{2n+1}}{2n + 1}$.
	\end{itemize}
\end{proposition}

\subsubsection{Calcul de Séries}
En vrac:
\begin{align*}
	1 - \frac{1}{3} + \frac{1}{5} - \cdots = \frac{\pi}{4}     \\
	1 + \frac{1}{4} + \frac{1}{9} + \cdots = \frac{\pi^{2}}{6} \\
	1 + \frac{1}{2} + \frac{1}{6} + \frac{1}{24} + \cdots = e
\end{align*}

\subsection{Séries Génératrices}
\begin{definition}
	Soit $u_{n}$ une suite. La \define{série génératrice} des $u_{n}$ est la série de terme général
	$u_{n}x^{n}$.
	Elle définit la \define{fonction génératrice} des $u_{n}$ sur son domaine de convergence.
\end{definition}

\begin{definition}
	La dérivée \define{formelle} de la série génératrice des $u_{n}$ est la série (génératrice) de terme
	général $(n + 1)u_{n + 1}x^{n}$ (des $(n + 1)u_{n + 1}$).
\end{definition}

\begin{proposition}
	Sur son domaine de convergence, la fonction génératrice de la dérivée formelle d'une série génératrice
	correspond avec la dérivée au sens classique de la fonction génératrice.
\end{proposition}

Ce résultat, cas particulier d'une version plus générale, permet de reconnaître aisément

\begin{proposition}
	Considérons deux variables aléatoires $(X, Y)$ uniformément distribuées sur $[0, 1]$.
	La probabilité que la partie entière du quotient $Y / X$ soit paire est $\frac{1}{2}(2 - \ln(2))$
\end{proposition}
\begin{proof}
	Observons que choisir $X, Y$ tel que précisé ci-dessus revient à choisir uniformément un point du carré
	unité $[0, 1] \times [0, 1]$.
	Notons de plus que $\lfloor Y / X \rfloor$ vaut exactement $2n \in \N$ si et seulement si le point $X, Y$
	est dans le triangle entre les points $(0, 0)$, $(1/n, 1)$ et $(1/(n + 1), 1)$.
	Calculer la probabilité souhaitée revient donc à calculer l'aire totale recouverte par ces triangles
	ainsi qu'à la partie $Y < X$ du carré.

	Ce triangle a pour aire
	\begin{equation*}
		\left(\frac{1}{n} - \frac{1}{n + 1}\right) \times 1 \times 1/2
	\end{equation*}

	On vérifie notamment que, ces triangles recouvrant la partie $Y > X$ du carré, la série de leurs
	aires vaut $1/2$.

	Pour notre problème, on ne s'intéresse qu'aux valeurs de $n$ paires, c'est à dire $n = 2k$.
	On a alors la série suivante pour calculer la probabilité recherchée
	\begin{align*}
		p & = \frac{1}{2}\left(\underbrace{1}_{\text{Région $Y < X$}} + \underbrace{\frac{1}{2} - \frac{1}{3}}_{\text{Région $3X > Y > 2X$}} + \underbrace{\frac{1}{4} - \frac{1}{5}}_{\text{Région $5X > Y > 4X$}} + \cdots\right) \\
		  & = \frac{1}{2} + \frac{1}{2}\sum_{k = 2}^{\infty} \frac{(-1)^{k + 1}}{k}                                                                                                                                                 \\
		  & = \frac{1}{2} +\frac{1}{2}\left( 1 - \underbrace{\sum_{k = 1}^{\infty}\frac{(-1)^{k + 1}}{k}}_{= \ln(1+x)(1)}\right)                                                                                                    \\
		  & = \frac{1}{2} + \frac{1}{2}(1 - \ln 2)
	\end{align*}
\end{proof}

\end{document}
