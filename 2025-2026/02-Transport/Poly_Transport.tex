\documentclass[info, math]{mpb-cours}

\title{Transport Optimal et Architecture}
\author{Matthieu Boyer}
\date{Cours TalENS 2 2025-2026}

\def\X{\mathcal{X}}

\begin{document}
\bettertitle

Ce polycopié ne doit pas être vu comme un remplacement au cours, mais simplement comme un résumé du contenu qui permet d'être sûr de ne rien manquer.

\section{Problème de Monge}
Gaspard Monge, ingénieur militaire de Napoléon, est chargé par ce dernier du problème suivant:
\begin{center}
	Étant donné des grognards portant des sacs de sable depuis $n$ camps, quelle est la manière optimale de
	construire $n$ murs à des endroits différents, sachant que les endroits sont à des distances plus ou
	moins grandes de chaque camp~?
\end{center}
Ce problème fait partie de la grande classe des problèmes d'optimisation, et est fondamental en théorie des
probabilités, puisqu'il est à la base de la théorie du transport optimal qui donne des manières de trouver
des distributions de probabilités "moyennes".

\subsection{Bases de Probabilités}
On ne rentrera pas ici dans les "vraies" bases de la théorie de la mesure, et notamment sur la notion de tribu.

\begin{definition}
	Une mesure sur un espace $\X$ muni d'une tribu ensemble $\Sigma$ de parties de $X$ est une application
	$\mu: \Sigma \to \R$ telle que:
	\begin{itemize}
		\item $\mu(\emptyset) = 0$
		\item $\mu(A \cup B) = \mu(A) + \mu(B) - \mu(A \cap B)$
		\item $\mu\left(\bigcup_{i = 0}^{\infty} A_{i}\right) = \sum_{i = 0}^{\infty} \mu(A_{i})$ si les $A_{i}$ sont deux
		      à deux disjoints
		\item $A \subseteq B \Rightarrow \mu(A) \leq \mu(B)$.
	\end{itemize}
	On dit que $\mu(\X)$ est la masse totale de $\mu$.
\end{definition}

Dans notre cas, on s'intéressera principalement aux cas $\abs{\X} \hookrightarrow \N$ et $X = \R^{d}$.
\begin{definition}
	Sur $\X$ au plus dénombrable, on a une mesure dite de comptage qui a chaque partie de $X$ associe son nombre d'éléments.

	Sur $\X = \R^{d}$ on a une mesure $\lambda$ dite mesure de Lebesgue qui à chaque partie de $X$ associe son volume.
	En particulier, $\lambda\left(\prod_{i} \left[a_{i}, b_{i}\right]\right) = \prod_{i} \abs{b_{i} - a_{i}}$.
\end{definition}

\begin{definition}
	Une mesure de probabilité est une mesure positive de masse $1$.
	Une mesure $\mu$ a densité par rapport à la mesure de Lebesgue s'il existe une fonction $p$ telle que
	\begin{equation*}
		\mu(A) = \int_{A} p(x)\d\lambda(x)
	\end{equation*}
\end{definition}
Le symbole intégrale $\int$ signifie ici que pour tout point $x$ dans $A$, on construit un petit pavé autour
de $A$, qu'on calcule son volume $\d\lambda(x)$ et qu'on le multiplie par la densité de $\mu$ en $x$ $p(x)$.

Autrement dit, dans le volume $\d\lambda(x)$ autour de $x$, il y a $p(x)$ particules. On trouve donc le
nombre total de particules dans $A$ en sommant les nombres de particules autour de tout point $x$ de $A$.

\begin{definition}
	Pour $x \in \R^{d}$, on définit le dirac en $x$ comme la mesure de probabilité $\delta_{x}$ qui vaut $1$ sur $\{x\}$ et $0$ ailleurs.

	Pour $\mu \in \R^{d}, \Sigma \in \M_{d, d}(\R)$, on définit la gaussienne centrée en $\mu$ de covariance
	$\Sigma$ par sa densité $p(x) = \exp\left(-\frac{1}{2}\Sigma^{-1}\left(x - \mu\right)^{2}\right)$.
\end{definition}

\begin{definition}
	Une variable aléatoire $X$ à valeurs dans $E$ est une fonction dite mesurable de $\X$ dans $E$.
	La probabilité que $X$ soit à valeurs dans $A \subseteq E$ est la mesure $\mu(X^{-1}(A))$.
\end{definition}

\end{document}
